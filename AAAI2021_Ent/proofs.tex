\documentclass[11pt,letterpaper]{article}
\usepackage{cogsys}
%\usepackage{cogsysapa} % deprecated!
% \usepackage{graphicx}
\usepackage[T1]{fontenc}
\usepackage{times}
\usepackage[pdftex]{graphicx} % use this when importing PDF files

% natbib required to produce author-year citations;
% apacite is not properly supported and may lead to errors
\usepackage{natbib}
\setlength{\bibsep}{0.75ex}

 % First page headings for accepted submissions.
\cogsysheading{X}{20XX}{1-6}{X/20XX}{X/20XX}
 % First page headings for poster submissions.
%\cogsysposterheading{First}{2012}{1-18}

\ShortHeadings{Epistemic Entrenchment and Counterfactual Reasoning in Automated Theory Change}
              {Xue Li, Alan Bundy, Eugene Philalithis}


\usepackage[many]{tcolorbox}
\usepackage{amsmath}
\usepackage{amssymb}
\usepackage{proof,amsthm, colordvi,changebar,ifthen}
\usepackage{tabularx}
\usepackage{multirow}
\usepackage{makecell}
\usepackage{enumitem}
\usepackage{mathtools}
\usepackage{slashbox}
\sloppypar % get rid of overrun a line



%----
\newtheorem{defn}{Definition}[section]
\newtheorem{ex}{Example}
\newtheorem{theorem}{Theorem}[section]
\newtheorem{heuristic}{Heuristic}
\newtheorem{exa}{Example}[section]
\newtheorem{assumption}{Assumption}[section]
\newtheorem{spec}{Specification}
\newtheorem{post}{Postulate}
\newcommand{\pf}[1]{\mathcal{#1}(\mathbb{PS})}
\newcommand{\ps}{\mathbb{PS}}
\newcommand{\dis}[1]{$\mathcal{d}(#1)$}
\newcommand{\red}[1]{\textcolor{red}{#1}}
\newcommand{\blue}[1]{\textcolor{blue}{#1}}
\newcommand{\green}[1]{\textcolor{green}{#1}}
\newcommand{\cyan}[1]{\textcolor{cyan}{#1}}
\newcommand{\brown}[1]{\textcolor{brown}{#1}}
\newcommand{\theory}{\mathbb{T}}
\newcommand{\norm}{\mathcal{N}}
\newcommand{\signature}{\mathcal{S}}
\newcommand{\struc}{\mathbb{PS}}
\newcommand{\num}{\mathcal{N}}
\newcommand{\newtheory}{\mathbb{T'}}
\newcommand{\uneq}{$\setminus${=}}
\newcommand{\true}{{\cal T}(\struc)}
\newcommand{\false}{{\cal F}(\struc)}
\newcommand{\incompa}{{\cal IC}(\theory,\struc)}
\newcommand{\insuffa}{{\cal IS}(\theory,\struc)}
\newcommand{\incompb}{{\cal IC}(\nu_k(\theory),\struc)}
\newcommand{\insuffb}{{\cal IS}(\nu_k(\theory),\struc)}

\newtcolorbox{mybox}[1]
{
  colframe = {blue},
  colback  = {grey},
  coltitle = {black},  
  title    = {#1},
}


\begin{document} 

\title{Epistemic Entrenchment and Counterfactual Reasoning in Automated Theory Change}
 
\author{Xue Li}{Xue.Shirley.Li@ed.ac.uk}
\author{Alan Bundy}{A.Bundy@ed.ac.uk}
\author{Eugene Philalithis}{E.Philalithis@ed.ac.uk}
\address{School of Informatics, University of Edinburgh, UK}
\vskip 0.2in
 
\begin{abstract}
This document contains the proofs of theorems in the paper: Epistemic Entrenchment and Counterfactual Reasoning in Automated Theory Change. The structure of this document contains the same sections as in the original paper, and proofs are given following their theorems for readability.
\end{abstract}

\section{Introduction}
\label{intro}
None.
\section{Axiom Entrenchment based on Preferred Structure}
\label{sec:ee:axiomEntrechment}


\begin{theorem}
For a fault-free theory $\theory$, we have:
\begin{align}
  \insuffa=\emptyset,\ &\iff &\forall p_{i} \in \pf{T}, \exists a, a\in \theory, c(a,\  p_{i}) \neq 0\\ 
  \incompa=\emptyset,\ &\iff& \forall a, a\in \theory,\ \mathcal{C}_{f}(a) = [\textbf{0}]
\end{align}
\end{theorem}

\section{Entrenchment of Preconditions}
\label{sec:ee:preconditions}


\begin{theorem}
When the impact of a precondition is the empty set, that precondition can be removed from the rule without changing the logical consequences of the theory.
\begin{equation}\label{equ:ref:theremPre}
  \mathcal{PI}(p_i(\vec{t_i})) = \emptyset \iff \mathcal{C}(\theory') = \mathcal{C}(\theory) 
\end{equation}
\end{theorem}


\begin{description}
\item[Proof:]
 If $\mathcal{PI}(p_i(\vec{t_i})) = \emptyset$, then according to Equation \ref{equ:preim}, $\not\exists \alpha, \alpha \in \mathcal{C}(\theory'),\ \alpha \notin \mathcal{C}(\theory)$. Thus, $\forall \alpha, \alpha \in \mathcal{C}(\theory'),\ \alpha \in \mathcal{C}(\theory)$, which equals to
 \begin{equation}\label{equ:ref:preT1}
     \mathcal{C}(\theory')\subseteq \mathcal{C}(\theory)
 \end{equation}
 
 Meanwhile, all proofs that contain $R$ remains after removing $p_i(\vec{t_i})$ from $R$ and $\theory'$ could have more theorems than $\theory$ as there is one precondition less in $R'$.
 \begin{equation}\label{equ:ref:preT2}
 \mathcal{C}(\theory)\subseteq \mathcal{C}(\theory')
 \end{equation}
 Combining \ref{equ:ref:preT1} and \ref{equ:ref:preT2}, Equation \ref{equ:ref:theremPre} is proved.
\end{description}


\begin{theorem}\label{the:def:precE}
The removal of a more entrenched precondition $p_1(\vec{t_1})$ is strictly dominated by the removal of a less entrenched precondition $p_2(\vec{t_2})$ from rule $R$.
\begin{align*}
    \mathcal{E}(p_1(\vec{t_1}), R) &\geq\mathcal{E}(p_2(\vec{t_2}), R) \iff   \\
    |\mathcal{IS}(\theory_1, \mathbb{PS}) | & \geq^*  |\mathcal{IS}(\theory_2, \mathbb{PS})|\wedge
   |\mathcal{IC}(\theory_1, \mathbb{PS})| \geq^* |\mathcal{IC}(\theory_2, \mathbb{PS})|
\end{align*}
where $\theory_1 = \theory \dot{-} R \dot{+} (R-p_1(\vec{t_1})),\ \theory_2 = \theory \dot{-} R \dot{+}. (R-p_2(\vec{t_2})) $.
\end{theorem}

\begin{proof}
 As $\mathcal{E}(p_1(\vec{t_1}), R) \geq\mathcal{E}(p_2(\vec{t_2}), R)$, bring Equation \ref{equ:eop1} and \ref{equ:eop2} to \ref{equ:ref:eopComp} so it can be concluded that:
 \begin{equation}\label{equ:ref:theo533}
 \begin{aligned}
     -|\insuffa - \mathcal{IS}(\theory_1, \mathbb{PS}) | &\geq^*
      -|\insuffa - \mathcal{IS}(\theory_2, \mathbb{PS}) | \wedge \\
     |\mathcal{IC}(\theory_1, \mathbb{PS}) - \incompa| &\geq^*
      |\mathcal{IC}(\theory_2, \mathbb{PS}) -  \incompa|
 \end{aligned}
 \end{equation}
$\forall \alpha$, if $\theory \vdash \alpha$, then $\theory_1 \vdash \alpha \wedge \theory_2 \vdash \alpha $, so it can be concluded that:
\begin{equation}\label{equ:ref:530}
\begin{aligned}
    \mathcal{IS}(\theory_1, \mathbb{PS}) \subseteq \insuffa \wedge \incompa \subseteq  \mathcal{IC}(\theory_1, \mathbb{PS}) \\
  \mathcal{IS}(\theory_2, \mathbb{PS}) \subseteq \insuffa  \wedge \incompa \subseteq \mathcal{IC}(\theory_2, \mathbb{PS}) 
\end{aligned}
 \end{equation}
Then Equation \ref{equ:ref:theo533} can be simplified as the following.
\begin{equation}\label{equ:ref:theo5332}
 \begin{aligned}
     -(|\insuffa| - |\mathcal{IS}(\theory_1, \mathbb{PS}) |) &\geq^*
      -(|\insuffa| - |\mathcal{IS}(\theory_2, \mathbb{PS}) | )\wedge \\
     |\mathcal{IC}(\theory_1, \mathbb{PS})| - |\incompa| &\geq^*
      |\mathcal{IC}(\theory_2, \mathbb{PS})| -  \incompa|
 \end{aligned}
 \end{equation}
By removing $\insuffa$ and $\incompa$ from both sides of the above inequalities, the following equation is derived.
 \begin{equation}
     |\mathcal{IS}(\theory_1, \mathbb{PS})|  \geq^*
      |\mathcal{IS}(\theory_2, \mathbb{PS})| \wedge 
     |\mathcal{IC}(\theory_1, \mathbb{PS})| \geq^*
      |\mathcal{IC}(\theory_2, \mathbb{PS})| 
 \end{equation}
 Thus, Theorem \ref{the:def:precE} is proved.
\end{proof}


\begin{theorem}\label{the:def:precE2}
The expansion of a more entrenched precondition $p_1(\vec{t_1})$ strictly dominates the expansion of a less entrenched precondition $p_2(\vec{t_2})$ to rule $R$.
\begin{align*}
    \mathcal{E}(p_1(\vec{t_1}), R_1) &\geq\mathcal{E}(p_2(\vec{t_2}), R_2) \iff   \\
    |\mathcal{IS}(\theory_1, \mathbb{PS}) | & \leq^*  |\mathcal{IS}(\theory_2, \mathbb{PS})|\wedge
   |\mathcal{IC}(\theory_1, \mathbb{PS})| \leq^* |\mathcal{IC}(\theory_2, \mathbb{PS})|
\end{align*}
where  $R_1 = R+p_1(\vec{t_1})$, $R_2 = R+p_2(\vec{t_2})$, $\theory_1 = \theory \dot{-} R \dot{+} R_1,\ \theory_2 = \theory \dot{-} R \dot{+} R_2 $.
\end{theorem}

\begin{proof}
Let $\theory$ and $\theory_1$in Theorem \ref{the:def:precE2} be $\theory'$ and $\theory$ in Equation \ref{equ:eop1} and \ref{equ:eop2} respectively. Then:
 \begin{align*}
    \mathcal{E}(p_1(\vec{t_1}), R_1)=(
     -|\mathcal{IS}(\theory_1, \mathbb{PS}) - \insuffa |,
     |\incompa -\mathcal{IC}(\theory_1, \mathbb{PS})|)
 \end{align*}
 Similarly, $\mathcal{E}(p_2(\vec{t_2}), R_1)$ can be written as the following.
  \begin{align*}
    \mathcal{E}(p_2(\vec{t_2}), R_2)=(
     -|\mathcal{IS}(\theory_2, \mathbb{PS}) -\insuffa|,
     |\incompa -\mathcal{IC}(\theory_2, \mathbb{PS})|)
 \end{align*}
 As $\mathcal{E}(p_1(\vec{t_1}), R_1) \geq\mathcal{E}(p_2(\vec{t_2}), R_2)$, bring the above equations to Equation \ref{equ:ref:eopComp}. 
 \begin{equation}\label{equ:ref:theo533244}
 \begin{aligned}
     -|\mathcal{IS}(\theory_1, \mathbb{PS}) - \insuffa |  &\geq^*
     -|\mathcal{IS}(\theory_2, \mathbb{PS}) - \insuffa | \wedge \\
     |\incompa -\mathcal{IC}(\theory_1, \mathbb{PS})| &\geq^*
      |\incompa -\mathcal{IC}(\theory_2, \mathbb{PS})| 
 \end{aligned}
 \end{equation}
$\forall \alpha$, if $\theory_1 \implies \alpha$, then $\theory \vdash \alpha$ and $\forall \beta$, if $\theory_2 \implies \beta$ then $\theory \vdash \beta$, so it can be concluded that:
\begin{equation}
\begin{aligned}
    \insuffa \subseteq \mathcal{IS}(\theory_1, \mathbb{PS})   \wedge  \mathcal{IC}(\theory_1, \mathbb{PS}) \subseteq \incompa  \\
  \insuffa \subseteq \mathcal{IS}(\theory_2, \mathbb{PS}) \wedge \mathcal{IC}(\theory_2, \mathbb{PS})  \subseteq \incompa 
\end{aligned}
 \end{equation}
Then Equation \ref{equ:ref:theo533244} can be simplified as the following.
\begin{equation}
 \begin{aligned}
     -(|\mathcal{IS}(\theory_1, \mathbb{PS}) |  - |\insuffa|) &\geq^*
     -(|\mathcal{IS}(\theory_2, \mathbb{PS}) | - |\insuffa| ) \wedge \\
     |\incompa| - |\mathcal{IC}(\theory_1, \mathbb{PS})|  &\geq^*
      |\incompa| - |\mathcal{IC}(\theory_2, \mathbb{PS})|
 \end{aligned}
 \end{equation}
 By removing $\insuffa$ and $\incompa$ from both sides of the above inequalities, the following is derived.
 \begin{equation}
     |\mathcal{IS}(\theory_1, \mathbb{PS})|  \red{\leq^*}
      |\mathcal{IS}(\theory_2, \mathbb{PS})| \wedge 
     |\mathcal{IC}(\theory_1, \mathbb{PS})| \red{\leq^*}
      |\mathcal{IC}(\theory_2, \mathbb{PS})| 
 \end{equation}
 Thus, Theorem \ref{the:def:precE2} is proved.
\end{proof}






\section{Entrenchment of Signature}
\label{sec:ee:sig}


\begin{theorem}
\label{therem:ref:theog}
If there is no path from predicate $p$ to predicate $q$ in the theory's theory graph, then assertions of $p$ cannot contribute to any proof of an assertion of $q$.
\end{theorem}
\begin{proof}
In Theorem \ref{therem:ref:theog}, no path from predicate $p$ to predicate $q$ in the theory's theory graph means that there is no rule connection between $p$ and $q$. Therefore, assertions of $p$ and $q$ are independent from each other, so that assertions of $p$ cannot contribute to any proof of an assertion of $q$. 
\end{proof}

\subsection{Predicate Entrenchment}



\begin{spec}\label{spec:pe}
Based on the  preferred distance, the important properties that predicate entrenchment $e(p)$ should have are as follows, where $p$, $p_{1}$ and $p_{2}$ are predicates, and $\mathbb{S}_{p}$ is the set of predicates which occur in $\ps$ while $\mathbb{S}_{t}$ is the set of predicates which occur in the theory but not in $\ps$.
\begin{enumerate}
\item $\forall p \in (\mathbb{S}_{t} \bigcup \mathbb{S}_{p}),\ e(p)$ has exactly one value.\newline
\textnormal{The entrenchment of a predicate should be just one value.}
\item $ 0 \leq e(p) \leq 1$. \newline
\textnormal{The range of an entrenchment should be [0,1], where 0 means that a predicate is not trusted at all and 1  represents that the predicate is most entrenched and fully trusted\footnote{In  Bayesian model and other probabilistic models \cite{gardenfors1988knowledge}, 1 is used as the value of the most entrenched. Although there was no explicit measurement of entrenchment, but only  certain properties that entrenchment should have in the previous literature, it is good to be coherent with their work, which is to employ 1 representing the most entrenched here.}.}

\item $\forall p_{2} \in \mathbb{S}_{p},\  e(p_{2})=1\ \bigwedge \  \forall p_{1}\in \mathbb{S}_{t},\  0< e(p_{1}) < 1$.\newline
\textnormal{Because $\ps$ is more trusted than the theory, a preferred predicate is most entrenched whose entrenchment is 1. Any predicate appearing only in the theory is believed in some sense, but less than a preferred predicate. Meanwhile, any predicate that occurs in the theory is considered to convey some information. Therefore, its entrenchment should be bigger than 0 but smaller than 1.  }
%\item If $p_{1} \in \mathbb{S}_{p},\ p_{2} \not\in \mathbb{S}_{p}$, then  $e(p_{1}) > e(p_{2})$. \newline The predicate $p_{1}$ occurs in $\ps$, so that it should be more entrenched than those which do not occur in $\ps$.

\item $\forall p_{1},p_{2}  \in \mathbb{S}_{t},\ e(p_{1}) > e(p_{2})$, iff $d_{p}(p_{1}) < d_{p}(p_{2})$.\newline
\textnormal{When neither predicate occurs in $\ps$, $p_{1}$  is more entrenched than $p_{2}$  if and only if $p_{1}$ is closer to preferred predicates in terms of its preferred distance. The smaller $d_{p}(p_{1})$ is, the more impact on $\ps$ changing  $p_{1}$ will have. As preferred  predicates should not be changed or effected, by assigning $p_{1}$  a bigger entrenchment value, it is less likely to be modified. }
\item If $p_{1} \in (\mathbb{S}_{t} \bigcup \mathbb{S}_{p})$, while $p_{2} \not\in (\mathbb{S}_{t} \bigcup \mathbb{S}_{p})$, then $ e(p_{1}) > e(p_{2})$.\newline
\textnormal{When $p_{2}$ is a predicate which neither appears in the theory nor $\ps$, it is an \emph{unknown predicate}. Unknown predicate $p_{2}$ is considered to be less entrenched than the ones occur in the theory or $\ps$.}
\end{enumerate}
\end{spec}


\begin{defn}[Predicate Entrenchment]\label{def:predEn}
The entrenchment of a non-isolated predicate $p_{1}$ is given in terms of its preferred distance and the maximum preferred distance of all non-isolated predicates in the theory graph:

\begin{equation}\label{equ:predEntr1}
\begin{multlined}
    e(p) =1- \frac{d_{p}(p)}{d_{pMax}+2} \\
 \text{where } d_{pMax} = \max_{\forall q \in \mathbb{S}_t}(d_{p}(q)),\  q \text{ is a non-isolated predicate}.
\end{multlined}
\end{equation} 
On the other hand, the entrenchment of an isolated predicate $p_{2}$ is normalised as follows:
\begin{equation}\label{equ:predEntr2}
\begin{multlined}
    e(p_{2}) =\frac{1}{d_{pMax}+2} \\
 \text{where } d_{pMax} = \max_{\forall q \in \mathbb{S}_t}(d_{p}(q)),\  q \text{ is a non-isolated predicate}.
\end{multlined}
\end{equation}
If a predicate $p_{3}$ is an unknown predicate\footnote{The zero entrenchment is defined to reflect that the existence of a predicate in the theory means that the predicate is trusted to have informational value.}, which appears neither in the theory nor in $\ps$, then its entrenchment is initialised as 0.
\begin{equation}\label{equ:predEntr3}
    e(p_{3}) =0
\end{equation}
\end{defn}


In the following paragraphs, we will prove that the defined measurement has the desired properties aforementioned. 
\begin{proof}

Based on Equation \ref{equ:predEntr2} and \ref{equ:predEntr3}, the entrenchment of an isolated predicate or an unknown predicate has one fixed value. Meanwhile, because the preferred distance of any non-isolated predicate has only one value, so that $d_{pMax}$ is a fixed value in a given theory. Consequently, the entrenchment of a non-isolated predicate has only one value too. In summary, property 1 is held by our measurement.

The preferred distance of a non-isolated predicate $p_{1}$ satisfies:
\begin{equation}\label{equ:p0}
  0 \leq d_{p}(p_{1}) \leq d_{pMax}
\end{equation}
According to Equation \ref{equ:predEntr1} and  \ref{equ:p0},  the entrenchment of a non-isolated predicate $p_{1}$ is in the range $[\frac{2}{d_{pMax}+2}, 1]$: 
\begin{equation}\label{equ:p1}
    \frac{2}{d_{pMax}+2} \leq e(p_{1})\leq  1 \\
 %  \forall p_{2} \in \mathbb{S}_{t},\  e(p)=0.
\end{equation}
Based on Equation \ref{equ:p1}, \ref{equ:predEntr2} and \ref{equ:predEntr3}, we can derive the following relation among the entrenchments of predicates in all different types, which proves  property 2, the first half of  property 3, the special case of  property 4 and  property 5.
\begin{equation}\label{equ:p2}
    0= e(p_{3}) < \frac{1}{d_{pMax}+2}= e(p_{2}) < \frac{2}{d_{pMax}+2} \leq e(p_{1})\leq  1 \\
 %  \forall p_{2} \in \mathbb{S}_{t},\  e(p)=0.
\end{equation}

 When $p$ is from $\ps$, then $d_{p}(p) = 0$. A preferred predicate $p$ is a non-isolated predicate, whose entrenchment can be calculated according to Equation \ref{equ:predEntr1}.
 
 \begin{equation}\label{equ:p3}
   If p \in \mathbb{S}_{p},\ then\  e(p) =1- \frac{d_{p}(p)}{d_{pMax}+2} = 1 - \frac{0}{d_{pMax}+2} =1
 \end{equation}
Combining Equation \ref{equ:p2} and \ref{equ:p3}, it can be seen that the property 3 is  proved. 

The difference of entrenchment between two non-isolated predicates $p_{11}$ and $p_{12}$ are given in the following equation.
 \begin{equation}\label{equ:p4}
   e(p_{11}) - e(p_{12}) =1- \frac{d_{p}(p_{11})}{d_{pMax}+2} - (1- \frac{d_{p}(p_{12})}{d_{pMax}+2}) = \frac{d_{p}(p_{12}) - d_{p}(p_{11})}{d_{pMax}+2}
 \end{equation}

Based on Equation \ref{equ:p0} and \ref{equ:p4}, property 4 can be proved:
 \begin{equation}\label{equ:p5}
 e(p_{11}) > e(p_{12}) \Leftrightarrow \frac{d_{p}(p_{12}) - d_{p}(p_{11})}{d_{pMax}+2} > 0 \Leftrightarrow d_{p}(p_{12}) > d_{p}(p_{11})
 \end{equation}

In summary, the defined measurement of entrenchment has all the desired properties given by Specification \ref{spec:pe}.

\end{proof}

\subsection{Argument Entrenchment}
\label{sec:ee:args}


\begin{defn}[The Entrenchment of an argument]
Let $ \mathcal{E}_{a}(p, n)$ be the entrenchment of the nth argument of predicate $p$, $n\leq arity(p)$:
\begin{equation}\label{equ:ref:aren}
    \mathcal{E}_{a}(p, n) =
    \begin{cases}
    |\mathcal{D}(p, n)|,\  p \ntriangleleft \mathbb{PS}\\
    Max(|\mathcal{D}(p', i)|)+1,\ p \triangleleft \mathbb{PS}, p' \ntriangleleft \mathbb{PS}, i\leq arity(p')
    \end{cases}
\end{equation}
where $p \triangleleft \mathbb{PS}$ represents that $\exists p(\vec{t}),\ p(\vec{t}) \in \pf{T} \bigvee p(\vec{t}) \in \pf{F}$, and the function $\mathcal{D}(p, n)$ returns the argument domain of the nth argument of predicate $p$ and $Max$ returns the maximum value.
\end{defn}


\begin{theorem}
The defined argument entrenchment is greater than one.
 \begin{align}
     \forall p, n,\ \mathcal{E}_{a}(p, n)\geq 1
 \end{align}
\end{theorem}
\begin{proof}
$\forall p, n,\ n \leq arity(p), |\mathcal{D}(p, n)|\geq 1$. So $Max(|\mathcal{D}(p, i)|)+1\geq 2$. \\
Therefore, $\forall p,n,\ \mathcal{E}_{a}(p, n)\geq 1$ based on Equation \ref{equ:ref:aren}.
\end{proof}

\begin{theorem}\label{the:ref:remarg}
A logical theory $\theory$ can be simplified by omitting the $n$th argument of predicate $p$ if $p$ is not in $\ps$ and the entrenchment is one $\mathcal{E}_{a}(p, n)=1$. If that argument is a variable in a rule, then weaken the remaining occurrences of that variable with the unique constant in the argument domain. Let the simplified theory be $\theory'$ and $p$ be the predicate of the omitted argument, then all theorems remains.
\begin{equation}
    \forall \alpha, \theory \vdash \alpha, \theory' \vdash \alpha*
\end{equation}
where $\alpha = \alpha*$ if $\alpha$ is not a theorem with $p$, otherwise, $\alpha*$ is different from $\alpha$ by lacking of the omitted argument.
\end{theorem}

\begin{proof}
As $\mathcal{E}_{a}(p, n) = |\mathcal{D}(p, n)| =  1$, let $\mathcal{D}(p, n) = \{c\}$. Assume that $R$ is the rule which contains $p(\vec{t})$ where $t_n = X$. Then in a proof which contains $R$, $X$ can only be bound to $c$. After deleting that argument from all propositions of $p$, the remaining $X$s in other propositions in $R$ are replaced by $c$. These changes amount to binding $X$ to $c$, so there is no effect to any proof which contains $R$. 

Therefore, all theorems of the theory remain the same except the ones in which $p$ lacks the omitted argument.
\end{proof}




\end{document} 


